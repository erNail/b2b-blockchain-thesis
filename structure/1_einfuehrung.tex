\pagenumbering{arabic}
\chapter{Einführung und Motivation}
\label{cha:einfuehrung}

Klassische \acs{B2B}-Anwendungen bringen Probleme hinsichtlich der Datenhaltung mit sich. Eine Option ist die Speicherung der Daten jeden Geschäftspartner selbst. Der Datenzugriff untereinander ist dann jedoch, aufgrund von aufwendig einzurichtenden Schnittstellen und uneinheitlichen Datenformaten, erschwert. Eine weitere Möglichkeit ist die Speicherung bei einem zentralen Unternehmen. Dieses hätte jedoch die Kontrolle über die Daten, womit alle anderen Parteien diesem vertrauen müssten. Diese Faktoren machen \acs{B2B}-Anwendungen für die Teilnehmer unattraktiv und erschweren die Entwicklung \cite{KorpelaDigitalSupplyChain2017}\cite{WustyouneedBlockchain2017}.

Um diese Probleme zu lösen, wird ein Prototyp einer dezentralen \acs{B2B}-Applikation, basierend auf der Blockchain-Technologie, entwickelt. Sie erlaubt es, dezentrale Systeme aufzubauen, in welchen sich die Parteien nicht vertrauen. Dabei werden die Daten bei jedem Teilnehmer des Blockchain-Netzwerks (im Folgenden nur noch Netzwerk genannt) selbst gespeichert. Trotzdem sind diese nicht lösch- oder manipulierbar, alle Transaktionen sind lückenlos nachvollziehbar und es besteht ein gemeinsamer Konsens über den Datenbestand \cite{CrosbyBlockChainTechnologyBitcoin2016}.

Bekannte Blockchain-Implementationen, wie Bitcoin oder Ethereum, bringen jedoch Probleme für den \acs{B2B}-Bereich mit sich. So sind alle Daten öffentlich einsehbar, der Transaktionsdurchsatz ist gering und die Konsensmechanismen sind unter bestimmten Umständen unsicher und resultieren in hohen Energieverbrauch \cite{Gramolidangerprivateblockchains2016}\cite{NakamotoBitcoinPeertoPeerElectronic2008}\cite{EthereumTeamEthereumWhitePaper2017}. 

Ziel dieser Arbeit ist es, die Probleme der Blockchain-Technologie für den \acs{B2B}-Bereich zu untersuchen und eine dezentrale \acs{B2B}-Anwendung als Proof-of-Concept zu entwickeln. Dazu werden zunächst die grundlegenden Konzepte der Blockchain-Technologie erläutert, um ein besseres Verständnis für die Vor- und Nachteile dieser zu erhalten. Anschließend werden die Probleme für \acs{B2B}-Anwendungen anhand der Anforderungen an diese genauer betrachtet und analysiert. Daraufhin erfolgt die Beschreibung der Anwendungsentwicklung. Zuletzt wird ein Fazit zur Lösung der Probleme und des entwickelten Systems gezogen.
