\pagenumbering{arabic}
\chapter{Einführung und Motivation}
\label{cha:einfuehrung}

Klassische B2B-Anwendungen bringen Probleme hinsichtlich der Datenhaltung mit sich. Eigene Daten können bei jeden Geschäftspartner selbst gespeichert werden, was jedoch den Zugriff auf diese, aufgrund von aufwendig einzurichtenden Schnittstellen und uneinheitlichen Datenformaten, erschwert. Eine weitere Möglichkeit ist die Speicherung bei einem zentralen Unternehmen. Dieses hätte jedoch die Kontrolle über die Daten, womit alle anderen Parteien diesem vertrauen müssten. Diese Faktoren machen B2B-Anwendungen für die Teilnehmer unattraktiv und erschweren die Entwicklung \cite{KorpelaDigitalSupplyChain2017}\cite{WustyouneedBlockchain2017}.

Um diese Probleme zu lösen wird ein Prototyp einer dezentralen B2B-Applikation, basierend auf der Blockchain-Technologie, entwickelt. Sie erlaubt es dezentrale Systeme aufzubauen, in welchen sich die Parteien nicht vertrauen. Alle Daten würden bei jedem Teilnehmer des Blockchain-Netzwerks (in der weiteren Arbeit nur noch Netzwerk genannt) gespeichert werden. Trotzdem sind diese nicht lösch- oder manipulierbar, alle Transaktionen sind lückenlos nachvollziehbar und es besteht ein gemeinsamer Konsens über den Datenbestand \cite{CrosbyBlockChainTechnologyBitcoin2016}.

Bekannte Blockchain-Implementationen, wie Bitcoin oder Ethereum, bringen jedoch Probleme mit sich, welche im B2B-Bereich von Nachteil sind. So sind alle Daten öffentlich einsehbar, der Transaktionsdurchsatz ist gering und die Konsensmechanismen sind unter bestimmten Umständen unsicher und resultieren in hohen Energieverbrauch \cite{Gramolidangerprivateblockchains2016}\cite{NakamotoBitcoinPeertoPeerElectronic2008}\cite{EthereumTeamEthereumWhitePaper2017}. 

Ziel dieser Arbeit ist es, die Probleme der Blockchain-Technologie für den B2B-Bereich zu analysieren und basierend auf den Ergebnissen eine dezentrale B2B-Anwendung als Proof-of-Concept zu entwickeln. Dazu werden zunächst die grundlegenden Konzepte der Blockchain-Technologie erläutert, um ein besseres Verständnis für die Vor- und Nachteile dieser zu erhalten. Anschließend werden die Probleme für B2B-Anwendungen anhand der Anforderungen an diese genauer betrachtet und analysiert. Daraufhin erfolgt die Beschreibung der Anwendungsentwicklung. Zuletzt wird ein Fazit zur Lösung der Probleme und des entwickelten Systems gezogen.
