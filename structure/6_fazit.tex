\chapter{Fazit und Ausblick}
\label{cha:fazit}

Die Blockchain-Technologie hat das Potenzial, dezentrale Anwendungen zu ermöglichen, in welchen Transaktionen nicht manipuliert oder gelöscht werden können. Öffentliche Blockchains bringen jedoch noch Probleme bezüglich Skalierbarkeit und Privatsphäre mit sich. Permissioned Blockchains können diese Probleme zu Teilen lösen, nehmen dafür aber eine stärkere Zentralisierung und weniger sichere Konsensmechanismen in Kauf. 

Es wurde ein Proof-of-Concept für einen dezentralen Wartungsmarkt entwickelt. Die Implementation von diesem mit der Blockchain bringt verschiedene Vorteile gegenüber klassischen B2B-Anwendungen mit sich. Die Teilnehmer müssen keiner zentral verwaltenden Instanz vertrauen und können ihre Daten trotzdem an einer einheitlichen Stelle zur Verfügung stellen. Es gilt jedoch noch verschiedene Dinge zu bedenken, Limitationen zu untersuchen und Funktionen zu verbessern. Aktuell kann keine eindeutige Aussage zu der Skalierbarkeit und Performance getroffen werden. Der maximal mögliche Transaktionsdurchsatz von Hyperledger Fabric müsste mit leistungsstärkeren Systemen untersucht werden, um eine Aussage darüber zu treffen, ob ein Wartungsmarkt mit mehreren tausend Teilnehmern und Maschinen bestehen kann. Diesbezüglich muss ebenfalls eine Analyse des \acs{PBFT} erfolgen, wenn es mehr als 64 Nodes gibt. Ebenfalls besteht die Frage, ob es für B2B-Anwendungen ausreichend ist, wenn \nicefrac{1}{3} an unvertrauenswürdigen Teilnehmern toleriert werden. Deshalb ist es wichtig, dass weitere Analysen von Hyperledger Fabric in der Zukunft erfolgen. Letztendlich steht fest, dass die Performance mit der Anzahl an Nutzern sinkt und Anwendungen deswegen auf einen bestimmten Transaktionsdurchsatz bzw. auf eine bestimmte Anzahl an Peer-Nodes beschränkt sind. 

Der entwickelte dezentrale Wartungsmarkt beinhaltet nur Features, welche für einen Proof-of-Concept benötigt werden. Er kann in diversen Bereichen erweitert werden. So können beispielsweise Remote-Support-Verträge eingeführt werden. Die Wartungsanbieter könnten, falls sie mit der Wartung keinen Erfolg haben, Hilfe über ihr Smartphone oder ihre Smart Glasses anfordern. Experten, welche Teilnehmer an der Blockchain sind, würden dann beispielsweise über ein Videogespräch den Support leisten. Ein weiteres interessantes Konzept ist ein Reputation System. Heutige Bewertungssysteme haben das Problem, dass falsche Bewertungen abgegeben werden, beispielsweise um Nutzer zu schädigen. Die Bewertung der Wartungsanbieter könnte jedoch automatisch über in der Blockchain bestehende Daten, wie zum Beispiel die benötigte Zeit für eine Wartung, erfolgen. Solche Systeme werden ebenfalls in den Arbeiten von Carboni \cite{CarboniFeedbackbasedReputation2015} und Dennis \cite{DennisRepblocknext2015} vorgestellt.

Ebenfalls bringen Hyperledger Fabric und Composer selbst Probleme mit sich. Die Konfiguration eines Netzwerks ist mit sehr viel Aufwand verbunden. Um im aktuellen Prototyp neue Teilnehmer hinzuzufügen, muss eine neue Konfiguration erstellt werden, wodurch auch alle davon abhängigen Shell-Skripte bearbeitet werden müssen. Ebenfalls gibt es nur einen mitgelieferten, nutzbaren Konsensmechanismus. Andere, wie PBFT, müssten selbst implementiert werden, was zusätzlichen Aufwand bedeutet. Das fehlende Data Sharing zwischen Channeln kann ebenfalls problematisch sein. Es ist nicht möglich, Daten vom Public Channel mit dem Private Channel zu teilen. So können private Transaktionen keine Referenz zu einem im Public Channel bestehenden Asset herstellen.

Ein letzter Punkt, welcher in Bezug zu Blockchain-Anwendungen erwähnt werden muss, ist, dass die Blockchain keine Korrektheit der Daten garantiert. So sagen beispielsweise Wüst et al., dass ``das Interface zwischen physischer und digitaler Welt'' ein Problem darstellt \cite{WustyouneedBlockchain2017}. So könnte beispielsweise beim Prototyp ein Wartungsanbieter durchgeführte Wartungsschritte dokumentieren, welche gar nicht erfolgt sind. Um dies zu lösen, könnte ein Sensor anhand verschiedener Daten überprüfen ob ein Wartungsschritt erfolgt ist. Dann muss allerdings diesen vertraut werden. Um Wartungsanbieter zu schädigen, könnte der Sensor vom Unternehmen manipuliert worden sein. Letztendlich garantiert die Blockchain nur, dass eingetragene Daten bzw. ausgeführte Transaktionen nicht manipulier- und löschbar sind sowie bei allen Teilnehmern gleich vorkommen. Um falschen Input zu verhindern, müssen andere Lösungen gefunden werden.

Auch wenn es noch Fragen zu klären gilt, zeigt der entwickelte Wartungsmarkt die Machbarkeit von Blockchain-Anwendungen für B2B. Die Probleme der Technologie sowie der Nutzen von Permissioned Blockchains müssen weiterhin untersucht werden. Ebenfalls ist wichtig, dass mehr Blockchain-Anwendungen entwickelt und eingesetzt werden. So werden die sinnvollen Use-Cases ergründet und die Technologie besser verstanden. Nur so kann sich die Blockchain im B2B-Bereich etablieren.


