%TODO: Ausreichend als Related Work ?
\chapter{Aktueller Stand der Technik}
\label{cha:stand-technik}

Die Blockchain wird seit 2008 erfolgreich für Kryptowährungen eingesetzt. Mit Ethereum wird das Konzept von Smart Contracts implementiert, womit es möglich ist eigene Programmlogik in der Blockchain abzubilden und so dezentrale Anwendungen zu entwickeln. Weitere weniger bekannte Blockchain-Implementationen für Public Blockchains sind Monero, Dashcoin und Litecoin \cite{BlockchainHubBlockchainsDistributedLedger}. 

Die Technologie bringt durch ihre Architektur jedoch auch Limitationen mit sich, weshalb sie nicht für alle Anwendungszwecke geeignet ist (Siehe \ref{cha:b2b-eval}. Die Probleme werden in vielen wissenschaftlichen Arbeiten analysiert und Lösungen für diese vorgeschlagen. Trotzdem bestehen gewisse Limitationen weiterhin \cite{ZhengBlockchainChallengesOpportunities2017}\cite{SwanBlockchainblueprintnew2015}\cite{SchererPerformanceScalabilityBlockchain2017}.

Permissioned Blockchains, wie Hyperledger Fabric oder Quorum bieten Vorteile gegenüber dem Public Blockchains. Sie bringen allerdings auch neue Herausforderungen mit sich. So muss z.B. eine Alternative zum PoW gefunden werden, um je nach Use-Case, Performance, Skalierbarkeit und Nichtangreifbarkeit sicher zu stellen \cite{LiScalablePrivateIndustrial2017}.

Dezentrale Märkte sind eine der am meisten mit Blockchain in Verbindung erwähnten Use-Cases, wie man an Quellen wie \cite{BenHamidaBlockchainEnterpriseOverview2017} und \cite{RavalDecentralizedApplicationsHarnessing2016} sehen kann.
Neben Konzepten für diese (Siehe \cite{KaiserDecentralizedPrivateMarketplace}), gibt es auch Live-Systeme, wie Syscoin \cite{SidhuSyscoinPeertoPeerElectronic2017}.

Dezentrale Wartungsmärkte hingegen werden nur in Verbindung mit der Supply Chain erwähnt, wie zum Beispiel in \cite{SoldatosWhatDoesBlockchain2017} oder \cite{GotzeLufthansaIndustrySolutions}. Implementationen oder Konzeptentwürfe konnten nicht gefunden werden.




