%----------------- KONFIGURATION ----------------- %
\pagestyle{empty} % enthalten keinerlei Kopf oder Fuß


\chapter*{Abstract} % (fold)
\label{cha:abtract}
Traditionelle B2B-Anwendungen mit multiplen Geschäftspartnern als Teilnehmer bringen verschiedene Probleme mit sich. Wenn jedes Unternehmen seine eigenen Daten speichert, erfolgt der Zugriff auf diese, für Kooperationspartner, aufwändig über Schnittstellen. Die Daten könnten sich auch bei einer einzelnen, eventuell nicht vertrauenswürdigen Instanz befinden, welche die Kontrolle über diese hat. Um dieses Problem zu lösen wird eine dezentrale B2B-Anwendung mittels der Blockchain-Technologie entwickelt. Die bekanntesten Implementationen dieser, wie Bitcoin und Ethereum, bringen jedoch Nachteile hinsichtlich Datenschutz, Sicherheit und Transaktionsdurchsatz mit sich, welche im B2B-Bereich nicht wünschenswert sind. Diese werden analysiert, um anschließend eine Aussage über den Nutzen von B2B-Blockchain-Anwendungen zu treffen, und sinnvoll den dezentralen Wartungsmarkt zu entwickeln.
