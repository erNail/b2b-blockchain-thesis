%----------------- KONFIGURATION ----------------- %
\pagestyle{empty} % enthalten keinerlei Kopf oder Fuß

%TODO: Abstract länger/kürzer ?
%TODO: Abstract Ende ändern
\chapter*{Zusammenfassung} % (fold)
\label{cha:abtract}
Traditionelle B2B-Anwendungen, mit multiplen Unternehmen als Teilnehmer, bringen Probleme bezüglich der Datenhaltung mit sich. So besteht die Möglichkeit, dass jedes Unternehmen Daten bei sich selber speichert. Dies führt jedoch zu zusätzlichen Aufwand, da Schnittstellen eingerichtet werden müssen, um Geschäftspartnern Zugriff auf relevante Daten zu gewähren. Eine weitere Option ist die Speicherung bei einer zentralen, verwaltenden Instanz. Jeder Geschäftspartner müsste dieser jedoch vertrauen, was die Anwendung unattraktiver macht. Eine Lösung für diese Probleme könnte die Blockchain-Technologie darstellen. Mit ihr ist eine verteilte Datenspeicherung unter nicht vertrauenswürdigen Teilnehmern möglich. Dabei stellt die Blockchain sicher, dass die Daten bei allen Geschäftspartnern synchron sind, und nicht manipuliert oder gelöscht werden können. Die Technologie bringt jedoch auch Schwierigkeiten mit sich, welche für B2B-Anwendungen unerwünscht sind. In Public Blockchains wie Bitcoin und Ethereum äußert sich dies anhand der Skalierbarkeit, Performance und Sicherheit. Permissioned Blockchains, wie Hyperledger Fabric, können diese Probleme zu Teilen lösen. Auf Grundlage dieser Erkenntnisse wurde ein Prototyp eines automatisierten dezentralen Wartungsmarktes als Proof-of-Concept entwickelt. In diesen können IoT-Geräte Wartungsbedarf erkennen und Wartungsanbieter darauf reagieren. Letztendlich lässt sich sagen, dass die Entwicklung mit Permissioned Blockchains noch Komplikationen mit sich bringt, und dass weitere zukünftige Evaluation der Technologie nötig ist, um eine endgültige Aussage über den Nutzen von Blockchains im B2B-Bereich zu treffen.
