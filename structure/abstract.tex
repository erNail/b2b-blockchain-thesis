%----------------- KONFIGURATION ----------------- %
\pagestyle{empty} % enthalten keinerlei Kopf oder Fuß


\chapter*{Abstract} % (fold)
\label{cha:abtract}
Traditionelle B2B-Anwendungen mit multiplen Geschäftspartnern als Teilnehmer bringen verschiedene Probleme mit sich. Wenn jedes Unternehmen seine eigenen Daten speichert, erfolgt der Zugriff auf diese, für Kooperationspartner, aufwändig über Schnittstellen. Die Daten könnten sich auch bei einer einzelnen, nicht vertrauenswürdigen Instanz befinden, welche die Kontrolle über diese hat. Um dieses Problem zu lösen kann die Blockchain-Technologie genutzt werden. Die bekanntesten Implementationen, Bitcoin und Ethereum, bringen jedoch Nachteile hinsichtlich Datenschutz, Sicherheit und Transaktionsdurchsatz mit sich, welche im B2B-Bereich nicht wünschenswert sind. Implementationen wie Hyperledger Fabric versprechen Lösungen für diese Probleme. Um dies zu evaluieren wird ein automatisierter sowie dezentralisierter Wartungsmarkt für IoT-Geräte entwickelt und untersucht.
