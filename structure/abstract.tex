%----------------- KONFIGURATION ----------------- %
\pagestyle{empty} % enthalten keinerlei Kopf oder Fuß


\chapter*{Abstract} % (fold)
\label{cha:abtract}
Zwischen mehreren Unternehmen soll ein Wartungsmarkt entstehen, in welchen IoT-Geräte Wartungsbedarf erkennen können und Wartungsanbieter darauf reagieren. Traditionelle B2B-Anwendungen mit multiplen Geschäftspartnern bringen jedoch Probleme bezüglich der Datenhaltung mit sich. So besteht die Möglichkeit, dass jedes Unternehmen Daten bei sich selber speichert. Dies führt jedoch zu zusätzlichen Aufwand, da Schnittstellen eingerichtet werden müssen, um Geschäftspartnern Zugriff auf relevante Daten zu gewähren. Eine weitere Option ist die Speicherung bei einer zentralen, verwaltenden Instanz. Jeder Geschäftspartner müsste dieser jedoch vertrauen, was die Anwendung unattraktiver macht. Eine Lösung für diese Probleme könnte die Blockchain-Technologie darstellen. Mit ihr ist eine verteilte Datenspeicherung unter nicht vertrauenswürdigen Teilnehmern möglich. Dabei stellt die Blockchain sicher, dass die Daten bei allen Geschäftspartnern synchron sind, und nicht mehr manipuliert oder gelöscht werden können. Die Technologie bringt jedoch auch Schwierigkeiten mit sich, welche für B2B-Anwendungen unerwünscht sind. Public Blockchains wie Bitcoin und Ethereum skalieren nicht, sind unsicher bei geringer Teilnehmerzahl und alle Daten sind für alle sichtbar. Permissioned Blockchains, wie Hyperledger Fabric, können diese Probleme zu Teilen lösen. Aufgrund dieser Erkenntnisse wird ein Prototyp eines dezentralen Wartungsmarktes als Proof-of-Concept entwickelt. Die Ergebniss sind letztendlich, dass die Entwicklung mit Permissioned Blockchains noch Komplikationen mit sich bringt, und dass weitere zukünftige Evaluation der Technologie nötig ist, um eine endgültige Aussage über den Nutzen von Blockchains im B2B-Bereich zu treffen.
