%----------------- KONFIGURATION ----------------- %
\pagestyle{empty} % enthalten keinerlei Kopf oder Fuß

\chapter*{Zusammenfassung} % (fold)
\addcontentsline{toc}{chapter}{Zusammenfassung}
\label{cha:abtract}

Traditionelle B2B-Anwendungen, mit multiplen Unternehmen als Teilnehmer, bringen Probleme bezüglich der Datenhaltung mit sich. Eine Option ist, dass jedes Unternehmen Daten bei sich selbst speichert. Dies führt jedoch zu zusätzlichen Aufwand, da Schnittstellen eingerichtet werden müssen, um Geschäftspartnern Zugriff auf relevante Daten zu gewähren. Eine weitere Option ist die Speicherung bei einer zentralen, verwaltenden Instanz. Jeder Geschäftspartner müsste dieser jedoch vertrauen, was die Anwendung unattraktiver macht. Eine Lösung für diese Probleme könnte die Blockchain-Technologie darstellen. Mit ihr ist eine verteilte Datenspeicherung unter nicht vertrauenswürdigen Teilnehmern möglich. Dabei stellt die Blockchain sicher, dass die Daten bei allen Geschäftspartnern synchron sind und nicht manipuliert oder gelöscht werden können. Die Technologie bringt jedoch auch Schwierigkeiten mit sich, welche für B2B-Anwendungen unerwünscht sind. In Public Blockchains wie Bitcoin und Ethereum äußert sich dies anhand der Skalierbarkeit, Performance und Sicherheit. Permissioned Blockchains, wie Hyperledger Fabric, können diese Probleme zu Teilen lösen. Nichtsdestotrotz sind darauf basierende Anwendungen auf eine bestimmte Performance oder Nutzerzahl limitiert. Auf Grundlage dieser Erkenntnisse wurde ein Prototyp eines automatisierten dezentralen Wartungsmarktes mit Hyperledger Fabric als Proof-of-Concept entwickelt. In diesem können IoT-Geräte Wartungsbedarf erkennen und Wartungsanbieter darauf reagieren. Beliebige Teilnehmer können an dem Markt teilnehmen, ohne dass Datenmanipulation durch eine Partei zu befürchten ist. Wartungen werden verfolgbar und unveränderbar dokumentiert, ohne dass Vertrauen zwischen den Parteien nötig ist.
