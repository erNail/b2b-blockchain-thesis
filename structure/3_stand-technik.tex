\chapter{Aktueller Stand der Technik}
\label{cha:stand-technik}

Die Blockchain-Technologie ist im Jahr 2008 entstanden. Jedoch befindet sie sich erst seit ca. 2016 in einem Hype, wie die Gartner Hype Cycle von 2016 und 2017 \cite{PanettaTopTrendsGartner2017}\cite{AmyGartner2016Hype2016} sowie die Google Trends zum Suchinteresse \cite{GoogleBlockchainGoogleTrends18} zeigen. Aufgrund des Hypes entstehen viele Ideen zu der möglichen Anwendung der Blockchain, darunter auch viele unnütze \cite{WustyouneedBlockchain2017}. Im Gartner Hype Cycle ist dies die ``Peak of Inflated Expectations''. Es bestehen unrealistische Erwartungen an die Technologie und die Anwendungszwecke sind noch nicht klar. 

Es finden sich wenig Blockchain-Anwendungen, welche bereits in Produktionsumgebungen eingesetzt werden. Dies liegt u.a. auch daran, dass es sich um eine relativ junge Technologie handelt. Viele Blockchain-Implementationen werden nur für Kryptowährungen eingesetzt. Dazu gehören z. B. Bitcoin, Monero, Dashcoin oder Litecoin \cite{BlockchainHubBlockchainsDistributedLedger}. Erst mit Ethereum wird das Konzept von Smart Contracts erfolgreich implementiert, womit es möglich ist eigene Programmlogik in der Ethereum-Blockchain abzubilden und so dezentrale Anwendungen zu entwickeln. So nutzt beispielsweise AXA diese für Flugversicherungen. Kommt es zu einer bestimmten Verspätung eines Fluges löst ein Smart Contract automatisch die Auszahlung an den Kunden aus \cite{BoerAXAnutztEthereumBlockchain2017}. Während es viele Ideen für dezentralisierte Anwendungen gibt, handelt es sich bei vielen nur um Prototypen.

Ein weiteres Problem an jungen Technologien ist, dass die Limitationen noch nicht komplett evaluiert und gelöst sind. Dies führt dazu, dass die Blockchain für bestimmte Anwendungen (noch) nicht geeignet ist. Die Probleme werden in diversen wissenschaftlichen Arbeiten analysiert und es werden Lösungen für diese vorgeschlagen (siehe \cite{ZhengBlockchainChallengesOpportunities2017}, \cite[S.~84]{SwanBlockchainblueprintnew2015} und \cite{SchererPerformanceScalabilityBlockchain2017}). Teilweise handelt es sich jedoch um Vorschläge, welche sich noch nicht im Einsatz befinden. Kapitel \ref{cha:b2b-eval} beschäftigt sich genauer mit den Problemen, welche die Blockchain mit sich bringt.

Für Permissioned Blockchains haben sich verschiedene Plattformen herausgebildet. So gibt es Multichain, Quorum, Hyperledger Fabric oder Hyperledger Burrow \cite{BenHamidaBlockchainEnterpriseOverview2017}. Diese sollen die Probleme der öffentlichen Blockchains lösen. Aber auch hier gilt es die Implementationen genauer zu analysieren, um eine Aussage über ihren Nutzen zu treffen. So setzen sich Arbeiten von \cite{BenHamidaBlockchainEnterpriseOverview2017}\cite{LiScalablePrivateIndustrial2017}, \cite{PongnumkulPerformanceAnalysisPrivate2017} und \cite{VukolicRethinkingPermissionedBlockchains2017} u.a. mit der Skalierbarkeit und Performance der Permissioned Blockchains auseinander.

Dezentrale Märkte sind eine der am meisten mit Blockchain in Verbindung erwähnten Use-Cases, was an Quellen wie \cite{BenHamidaBlockchainEnterpriseOverview2017} und \cite[S.~33 ff.]{RavalDecentralizedApplicationsHarnessing2016} ersichtlich wird.
Neben Konzepten für diese (siehe \cite{KaiserDecentralizedPrivateMarketplace2017}), gibt es auch Live-Systeme, wie Syscoin \cite{SidhuSyscoinPeertoPeerElectronic2017}.

Dezentrale Wartungsmärkte hingegen werden nur in Verbindung mit der Supply Chain erwähnt, wie zum Beispiel in \cite{SoldatosWhatDoesBlockchain2017} oder \cite{GotzeLufthansaIndustrySolutions}. Implementationen oder Konzeptentwürfe konnten nicht gefunden werden.