\chapter{Dezentraler Wartungsmarkt - Prototyp}
\label{cha:wartungsmarkt-impl}

Das vorherige Kapitel hat gezeigt, dass es möglich ist Permissioned Blockchains aufzubauen, welche die im Kapitel \ref{cha:concept} erwähnten Anforderungen erfüllen. So ist es möglich über z.B. Hyperledger Fabric einen Transaktionsdurchsatz von mindestens 350 TPS zu erreichen. Weiterhin gibt es Konsensmechanismen, welche 1/3 an unvertrauenswürdigen Nodes tolerieren und einen Transaktionsdurchsatz von ungefähr 4500 TPS, je nach Teilnehmeranzahl, erzielen. Zusätzlich erlauben Technologien wie Hyperledger Fabric und Quorum das ausführen von privaten Transaktionen. Im Folgenden Kapitel wird eine Blockchain-Technologie ausgewählt, und der dezentrale Wartungsmarkt anhand dieser implementiert.  
  
%TODO: Erwähnen das registrieren von neuen Teilnehmern und Maschinen kein Teil dieser Arbeit ist, nur kurz Möglichkeit dazu erklären
\section{Technologieauswahl}

%TODO: S.2 WustYouNeedBlockchain: Decentralized Blockchain, but central entity decides about participants
%TODO: S.4 WustYouNeed: Skuchain based on Hyperledger Fabric
%TODO: S.1 HyperledgerWhitepaper: For business networks
%TODO: S.1 HyperledgerWhitepaper: It implements industry requirements (Performance, identitied, private transactions, pluggable consensus)
%TODO: S.2 HyperledgerWhitepaper: Most Use-Cases are mmostly expected to require permissioned blockchains, cryptocurrencies do not directly support this.
%TODO: S.4-5 HyperledgerWhitepaper: Hyperledger Whitepaper is supposed to be as modular and extendable as possible --> More Use-Cases, easy working, mehr Weiterentwicklung
%TODO: S.6 HyperledgerWhitepaper Private Transactions
%TODO: S.4 BenHamida Tabelle Technologien
%TODO: S.1 Vukolic Nachteile vpn Kadena, Tendermint, Chain
%TODO: S.3 Vukolic: Pluggable Consens is important

\begin{itemize}
    \item Welche Blockchain wird genutzt um die Anforderungen zu erfüllen ?
\end{itemize}


\label{sec:hyperledger-fabric-composer}
\section{Hyperledger Fabric und Composer}

\subsection{Hyperledger Fabric}

%TODO: S.15 HyperledgerWhitepaper: Services (Identity, Policy, etc.)
%TODO: S.17 HyperledgerWhitepaper: Internal Data Structurem, Large Documents not stored off-chain, but their hashes are stored as part of the transactions-->Integrity is kept
%TODO: S.14 Scherer: Hyperledger Fabric Definition, Identities, Private Channels
%TODO: S.14 Scherer: Hyperledger State Database
%TODO: S.14 Scherer: Architecture COmponents with peer nodes, ordering nodes, client applications
%TODO: S.15 Scherer: Chaincode Explanation, Deployment, Interaction, Transaction Flow
%TODO: S.3-4 Vukolic: Seperate smart contracts and consensus
%TODO: Consensus in Hyperledger Channels


\subsection{Hyperledger Composer}
%TODO: Hinweis auf frühe Version und kontinuierliche Entwicklung
%TODO: HyperledgerComposerTeamIntroduction: Composer Introduction

\section{Modell}
\begin{itemize}
    \item Architekturen, Sequenzdiagramme, Workflows etc.
    \item Datenmodell (Participants, Assets, Transaktionen)
    \item Netzwerkarchitektur
\end{itemize}

\section{Gerätesimulation durch Bosch XDK}
\begin{itemize}
    \item Simulation eines IOT-Geräts durch einen Bosch XDK
\end{itemize}

\section{Programmlogik}
\begin{itemize}
    \item Funktion der Transaktionen
\end{itemize}

\section{Benutzeroberflächen}
\begin{itemize}
    \item UIs für die Interaktion mit der Blockchain
\end{itemize}

\section{Auswahl des Konsensmechanismus}
%TODO:


%TODO: Was ist wenn Geräte falsch funktionieren ? Oder vom Hersteller manipuliert sind ?
%TODO: Wie verhindern, dass Anbieter Geräte registrieren, die nicht existieren ?
%TODO: ACL Rules auf einzelne Properties anwenden
%TODO: Kontinuierliches Loggen des Gerätestatus --> Dadurch vielleicht besseres Erkennen ob Geräte Hardware fehlerhaft/manipuliert ist ?
%TODO: Evaluierung als extra Kapitel ?

%TODO: S.7 WustYouNeed: The interface between the physical and digital world are the problem. Sensors need to be trusted
\section{Evaluierung}
\begin{itemize}
    \item Analyse des Systems in Bezug auf Anforderungen und Blockchain-Probleme
    \item Skalierbarkeit
    \item Transaktionsdurchsatz
    \item Limitationen der Applikation (Neue Teilnehmer hinufügen, Cross Channel Data Sharing, Transaktionen pro Sekunde, Sicherheit etc.)
\end{itemize}



