\chapter{Evaluierung Permissioned Blockchains für B2B}
\label{cha:b2b-eval}


\section{Skalierbarkeit}
\label{sec:scalability-eval}
%TODO: S.12 ZhengBlockchainChallenges GHOST-Explanation 
%TODO: S.2 WustYouNeed Vergleich Visa Bitcoin

%TODO: Erwähnen, wonach sich die Gebühr richtet ?
...Im Falle von Bitcoin gilt eine Transaktion als bestätigt, sobald Sie in einem Block vorkommt und 6 Nachfolger hat. Zum einen entsteht eine Wartezeit dadurch, dass 6 mal ein Proof-of-Work erbracht werden muss, und zum anderen priorisieren Miner die Transaktionen nach der Transaktionsgebühr. Desto höher diese ist, destro größer ist die ausgezahlte Belohnung \cite{BuchkoHowLongBitcoin2017}.
\begin{itemize}
    \item Transaktionsdurchsatz --> Vor allem wichtig bei IoT-Geräten
    \item Bestätigung von Transaktionen ?
    \item Datenmenge und Redundanz
\end{itemize}

\label{subsec:eval-konsens}
\section{Konsensmechanismen}
%TODO: S.8 ZhengBlockchainChallenges: Consensus Algorithm Byzantine Generals Erklärung
%TODO: S.10-11 ZhengBlockchainChallenges: Consensus Comparison
%TODO: S.4 ZhengBlockchainChallenges: Proof of Stake --> Rich get richer
%TODO: S.7 WustYouNeed: PBFT Consensus is enough for permissioned blockchains
%TODO: S.117 CromanScaling: PBFT could be enough for permissioned blockchains


\begin{itemize}
    \item Nachteile Proof-of-Work
    \item Analyse anderer Konsensmechanismen
\end{itemize}

\section{Datenschutz}
%TODO: S.4 ZhengBlockchainChallenges: Privacy Leakage and IP Tracking
%TODO: S.1 WustYouNeedBlockchain: Privacy in permissionless chains (zerocash)
%TODO: S.2 WustYouNeedBlockchain: Tensio between Transparency and Privacy

\begin{itemize}
    \item Müssen alle Daten öffentlich verfügbar sein ?
    \item Private Transaktionen realisieren
\end{itemize}


