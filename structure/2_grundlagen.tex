\pagenumbering{arabic}
\chapter{Blockchain-Grundlagen}
\label{cha:grundlagen}

\section{Funktionsweise}
Die Funktionsweise der Blockchain wird hauptsächlich an Bitcoin erklärt. Als erste Blockchain-Anwendung \cite{ZhengBlockchainChallengesOpportunities2017} und aufgrund der relativ geringen Komplexität liefert es die Grundlage für die Funktion der Technologie. Andere Implementationen, wie Ethereum oder Ripple, funktionieren nach dem gleichen Prinzip.

\subsection{Allgemein}
Wenn der Begriff ``Die Blockchain'' auftaucht, ist damit meistens die Blockchain-Technologie gemeint. Es gibt nicht nur eine global bestehende Blockchain und auch nicht nur eine Implementation der Technologie, was man an Bitcoin oder Ethereum sehen kann.

Allgemein kann man die Blockchain als Datenstruktur bezeichnen, welche durch kryptographische Verfahren verteilt bei den Teilnehmern der Anwendung, nicht löschbar und unmanipulierbar gespeichert werden kann.
\begin{itemize}  
  \item Verkettung und Blöcke
  \item Verteiltheit
  \item Funktion des Netzwerks
  \item Validität von Transaktionen
\end{itemize}

\subsection{Konsensmechaniken}
\begin{itemize}
  \item Funktion (Wozu Konsensmechaniken ?)
  \item Beispiel: Proof-of-Work
  \item Forking
\end{itemize}

\subsection{Angreifbarkeit}
\begin{itemize}
  \item Wie sicher ist die Blockchain ?
\end{itemize}

\subsection{Blockchaintypen}
\begin{itemize}
  \item Public/Permissioned/Private Blockchains
\end{itemize}

\subsection{Exemplarische Anwendungsfälle}
\begin{itemize}
    \item Wozu kann die Blockchain genutzt werden ?
\end{itemize}

\subsection{Probleme für den B2B-Bereich}
\begin{itemize}
    \item Welche Nachteile hat die Blockchain im B2B-Bereich ?
\end{itemize}
