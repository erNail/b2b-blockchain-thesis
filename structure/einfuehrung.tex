\pagenumbering{arabic}
\chapter{Einführung und Motivation}
\label{cha:einfuehrung}

\begin{itemize}
    \item Entwicklung eines dezentralen Wartungsmarktes
    \item Wozu ein dezentraler Wartungsmarkt ? Klassische B2B-Anwendungen und deren Probleme beschreiben (Viele Kooperationsverhandlungen, evtl. herausbilden einer neuen kontrollierenden Instanz, Sensible Daten bei anderen Unternehmen) 
  
     \begin{itemize}
      \item Daten bei jeden Unternehmen, aufwändiger Zugriff, unterschiedliche Datenformate, etc.
      \item Alle Daten zentral bei einer Instanz (Evtl. auch sensible Daten)
      \item Partner schrecken wegen Nichtvertrauen zurück
      \item Zugriff aufwändig über APIs
    \end{itemize}
    \item Lösung: Blockchain, kurz Kernelemente nennen
    \item Probleme bekannter Blockchain-Technologien für B2B-Zwecke nennen
    \item Technologien und Arten(Permissioned/Public) erwähnen, welche diese Probleme lösen sollen
    \item Theoretische und praktische Analyse durch Implementation eines dezentralisierten Wartungsmarktes (Genauer beschrieben), welcher die Probleme von klassischen B2B-Anwendungen löst.
    \item Ziel der Arbeit: Implementierung des dezentrales Wartungsmarktes beschreiben, und in Zuge dessen den Nutzen der Blockchain-Technologie im B2B-Bereich evaluieren sowie die Anwendung darauf untersuchen
    \item Vorgehen:
    \begin{itemize}
      \item Blockchain Grundlagen erklären
      \item Dezentralen Wartungsmarkt und die Anforderungen an diesen erklären
      \item Nachteile der Blockchain und Auswirkung auf B2B-Bereich nennen 
      \item Theoretische Analyse, ob die Probleme gelöst werden können
      \item Implementierung des dezentralen Wartungsmarktes beschreiben
      \item Fazit/Ausblick zur Lösung der Probleme und des entwickelten Systems geben
    \end{itemize}
  \end{itemize}

  \cite{WustyouneedBlockchain2017}
